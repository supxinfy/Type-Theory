\documentclass[openany]{book}

\usepackage[utf8]{inputenc}
\usepackage[T1, T2A]{fontenc}
\usepackage[greek.polytonic, russian]{babel}

\usepackage[a4paper, total={6in, 9in}]{geometry}

\usepackage{amssymb, amsthm, amsmath, amsfonts, bbm, float, listings, stmaryrd}
\usepackage[dvipsnames]{xcolor}

\usepackage{bbm}
\usepackage{hyperref}
\hypersetup{
    colorlinks = true,
    pdftitle={Теория типов},
    pdfsubject={Теория типов},
    pdfauthor={supxinfy},
    pdfkeywords={ 
        Теория типов
    },
    citecolor=ForestGreen,
}
\usepackage{graphicx}
\usepackage{nicefrac}
\usepackage{microtype}
\usepackage{cleveref}
\usepackage{url}
\usepackage{tikz}

\usetikzlibrary{positioning, shapes.geometric}
\usetikzlibrary{graphs} 

\tikzset{fontscale/.style = {font=\relsize{#1}}
}

\usepackage{alphalph}
\usepackage{mathrsfs} 
\usepackage{upgreek} 
\usepackage{bussproofs}

\theoremstyle{plain}
\newtheorem{thm}{Теорема}[section]
\newtheorem{lem}[thm]{Лемма}
\newtheorem{prop}[thm]{Предложение}
\newtheorem{cor}{Следствие}	
\newtheorem{note}{Замечание}[section]

\theoremstyle{definition}
\newtheorem{defn}{Определение}[section]
\newtheorem{xrc}{Упражнение}[]
\newtheorem{eg}{Пример}[]

\newenvironment{sol}
{\begin{proof}[Решение]}
{\end{proof}}

\newcommand{\N}{\mathbb{N}}
\newcommand{\rec}{\mathrm{rec}}
\newcommand{\nat}{\mathsf{Nat}}

\date{\today}
\author{Шубин Владислав}
\title{Теория типов: Конспект}

\begin{document}

\maketitle
\tableofcontents

\chapter{Введение}
Теория типов представляет собой новое направление в логике, изучающее системы типов. С математической точки зрения существует два пути происхождения теории типов. Первый из них есть анализ математического текста, например из книги Шафаревича \cite{Shaph} мы видим следующее «для \(y \in Y\) и \(x \in f^{-1}(y)\) мы получаем уравнение \[t_i(x)^k + a_1(y)t_i(x)^{k-1}+\dots+a_k(y) = 0.\text{»}\]В этом контексте мы хотим анализировать подобные высказывания, которые понимаются, обычно, интуитивно. Например, здесь бессмысленно было бы сказать \(x \in \Bbbk\), по причине того, что \(x\) --- это точка аффинной схемы, но не элемент поля. В этом смысле они обладают разными «типами». Человек, знакомый с алгебраической геометрией, понимает, что \(k \in \N\), \(t_i : X \to \Bbbk\), \(a_i : Y \to \Bbbk\), где \(\Bbbk\) есть поле.

Подобные выражения недоступны в языке логики первого порядка и для того, чтобы оперировать с такой «математической грамматикой» нам понадобится теория типов

В теории типов мы, подобно символу \(\in\), используем символ \(:\), означающий «иметь тип». Например, \(x : A\) означает «\(x\) имеет тип \(A\)». Пусть у нас есть несколько типов \(x : A, y : B\), которые мы отделяем запятой. Если из них возможен вывод, мы его будем называть суждением, а посылку контекстом. То есть \[\underbrace{x : A, y : B}_{\text{контекст}} \vdash \underbrace{t : C}_{\text{суждение}}.\]

Кроме чисто математической точки зрения мы имеем довольно важную практическую сторону этой теории --- это языки программирования. С точки зрения компьютера (если у него, конечно, может быть какая-либо точка зрения) любые данные представляют из себя кусок бинарного кода. Картинки, программы, музыка --- все это есть лишь последовательность нулей и единиц. Поэтому в ранних языках программирования существовала проблема типизации. Для примера возьмем Си:
\begin{lstlisting}[language=C]
    void rev(char *str, int len){
        int start = 0;
        int end = len - 1;
        while(start < end) {
            char tmp = str[start];
            str[start] = str[end];
            str[end] = tmp;
            end--;
            start++;
        }
    }
\end{lstlisting}
Современный язык Си используем систему типов, в которой тип \texttt{char} представляет символьный тип. Однако, нет никакой существенной разницы между типами \texttt{char} и \texttt{short}, например (второй тип представляет однобайтовое число). Поэтому эту функцию можно применить и к массиву чисел, что создает существенные проблемы при разработке сложного ПО.

Из-за этой проблемы сформировалось, своего рода, ad hoc решение, называемое строгой типизацией. 

\begin{note}
    В математике любой объект можно представить как некоторое множество, однако это не лишает теорию типов приложимости. Нам важно, что натуральное число --- это не просто множество, а что-то новое, обладающие свойствами «числа».
\end{note}

\begin{eg}
    Можно представить следующий пример вычислений с типами 
    \begin{prooftree}
        \AxiomC{\(m, n: \N \vdash 2 \cdot m + n : \N\)}
        \AxiomC{\(\vdash 2 : \N\)}
        \BinaryInfC{\(\vdash 2 \cdot 5 + 3 : \N\)}
    \end{prooftree}
\end{eg}

\section{Простая теория типов}

Введем нашу первую модель теории типов формально. 
Типы мы определим как \[A := 1 \, | \, A_1 \times A_2 \, | \, \nat,\] то есть свободно порожденное множество с элементами \(1 \in \mathsf{Ty}\), \(\nat \in \mathsf{Ty}\), \(\times: \mathsf{Ty}^2 \to \mathsf{Ty}\). Под \(1\) мы имеем ввиду унарный тип. И, как и во всякой теории, нам понадобятся термы \[t := x \, | \, \langle \rangle \, | \, \langle t_1, t_2 \rangle \, | \, p_1(t) \, | \, p_2(t) \, | \, 0 \, | \, S(t) \, | \, \rec(...),\] где \(\rec\) --- рекурсия, чьи аргументы мы уточним позже.

Далее нам понадобится контекст, определяемый как \(\Gamma := \langle \rangle \, | \, \Gamma, x : A\), то есть контексты суть конечные списки переменных с типами (например, \(x_1 : A_1, \dots, x_n : A_n\)).

Теперь займемся суждением о типе --- отношении \(\Gamma \vdash t : A\) или «в контексте \(\Gamma\) переменная \(t\) имеет тип \(A\)» что аналогично «выводимости» в логике первого порядка.

\subsection{Аксиомы}
Первая часть правил будет относиться к так называемым структурным правилам. Под \(i\) мы будем иметь в виду любой тип в контексте.

Перечислим их:
\begin{itemize}
    \item (Axiom, Identity, Assumption)
        \begin{prooftree}
            \AxiomC{}
            \UnaryInfC{\(x_1 : A_1, \dots, x_n : A_n \vdash x_i : A_i\)}
        \end{prooftree}

    \item Подстановка (Substitution)
        \begin{prooftree}
            \AxiomC{\(\Delta \vdash s_i : A_i \quad \Gamma \vdash t : C\)}
            \UnaryInfC{\(\Delta \vdash t[s_i / x_i] : C\)}
        \end{prooftree}
\end{itemize}

Аксиому подстановки можно вывести из следующих аксиом:
\begin{itemize}
    \itemОслабление (weakening):
    \begin{prooftree}
        \AxiomC{\(\Gamma \vdash t : A\)}
        \UnaryInfC{\(\Gamma, x : B \vdash t : A\)}
    \end{prooftree}
    \itemЗамена (exchange):
    \begin{prooftree}
        \AxiomC{\(\Delta \vdash s_i : A_i \quad \Gamma \vdash t : C\)}
        \UnaryInfC{\(\Gamma, x : A, y : B \quad \Delta \vdash t : C\)}
    \end{prooftree}
    \itemЗамена одной переменной:
    \begin{prooftree}
        \AxiomC{\(\Gamma, x : A \vdash t : B\)}
        \AxiomC{\(\Gamma \vdash a : A\)}
        \BinaryInfC{\(\Gamma \vdash t [a/x] : B\)}
    \end{prooftree}
\end{itemize}

\begin{eg}
    Пример работы аксиомы подстановки
    \begin{prooftree}
        \AxiomC{\(y : \N \vdash y \cdot y : \N \quad x_1, x_2 : \N \vdash x_1 + x_2 : \N\)}
        \UnaryInfC{\(x_1, x_2 : \N \vdash x_1 \cdot x_1 + x_2 \cdot x_2 : \N\)}
    \end{prooftree}
\end{eg}

Наша система все еще не является теорией типов, поскольку у нас нет правил вывода типов. Давайте их введем:
\begin{itemize}
    \itemПроизведение
    \begin{prooftree}
        \AxiomC{\(\Gamma \vdash t_1 : A_1, \Gamma \vdash t_2 : A_2\)}
        \UnaryInfC{\(\Gamma \vdash \langle t_1, t_2\rangle : A_1 \times A_2\)}
    \end{prooftree}
    \itemПроекция
    \begin{prooftree}
        \AxiomC{\(\Gamma \vdash t : A_1 \times A_2\)}
        \UnaryInfC{\(\Gamma \vdash p_i(t) : A_i\)}
    \end{prooftree}
    \item\(0\)
        \begin{prooftree}
            \AxiomC{\(\)}
            \UnaryInfC{\(\Gamma \vdash 0 : \nat\)}
        \end{prooftree}
    \item\(S(n)\)
        \begin{prooftree}
            \AxiomC{\(\Gamma \vdash n : \nat\)}
            \UnaryInfC{\(\Gamma \vdash S(n) : \nat\)}
        \end{prooftree}
    \item\(1\)
        \begin{prooftree}
            \AxiomC{\(\)}
            \UnaryInfC{\(\Gamma \vdash \langle \rangle : 1\)}
        \end{prooftree}
    \item\(\rec\)
        \begin{prooftree}
            \AxiomC{\(\Gamma \vdash n : \nat\)}
            \AxiomC{\(\Gamma \vdash t_0 : C\)}
            \AxiomC{\(\Gamma, x : \nat, y : C \vdash t_s (x, y) : C\)}
            \TrinaryInfC{\(\Gamma \vdash \rec(t_0; (x, y, t_s); n) : C\)}
        \end{prooftree}
\end{itemize}

Чтобы определить рекурсию на \(n \in \N\) нам нужна база \(n = 0\) и индуктивный переход \(n = S(m)\).

Теперь мы ввели все, что нужно для теории типов. Однако, стоит учитывать один нюанс.

\subsubsection{Аннотации}

Давайте посмотрим внимательно на аксиому проекции
\begin{prooftree}
    \AxiomC{\(\Gamma \vdash t : A_1 \times A_2\)}
    \UnaryInfC{\(\Gamma \vdash p_i(t) : A_i\)}
\end{prooftree}
Из этой аксиомы нельзя точно понять, на каком типе определена функция \(p_i\), но мы видим этот тип из контекста. В теории типов существует аннотирование, которое приписывает на каком типе определена функция. С ней наша аксиома должна выглядеть как 
\begin{prooftree}
    \AxiomC{\(\Gamma \vdash t : A_1 \times A_2\)}
    \UnaryInfC{\(\Gamma \vdash p^{A_1, \, A_2}_i(t) : A_i\)}
\end{prooftree}
Аннотация присутствует всегда, но для читаемости ее не выписывают. Нам также необходимо аннотировать аксиомы пары, поскольку функция пары для каждого типа своя:
\begin{prooftree}
    \AxiomC{\(\Gamma \vdash t_1 : A_1, \Gamma \vdash t_2 : A_2\)}
    \UnaryInfC{\(\Gamma \vdash \langle t_1, t_2\rangle^{A_1, \, A_2} : A_1 \times A_2\)}
\end{prooftree}
Для рекурсии нужно аннотировать саму функцию \(\rec\):   
\begin{prooftree}
    \AxiomC{\(\Gamma \vdash n : \nat\)}
    \AxiomC{\(\Gamma \vdash t_0 : C\)}
    \AxiomC{\(\Gamma, x : \nat, y : C \vdash t_s (x, y) : C\)}
    \TrinaryInfC{\(\Gamma \vdash \rec^C(t_0; (x, y, t_s); n) : C\)}
\end{prooftree} 

\subsubsection{Упрощение выражений}

Мы уже ввели теорию типов. Для нее не обязательны аксиомы для упрощения выражений, но давайте их введем для удобства.

\begin{prooftree}
    \AxiomC{\(\Gamma \vdash t_1 : A_1, \Gamma \vdash t_2 : A_2\)}
    \UnaryInfC{\(\Gamma \vdash p_i \langle t_1,\, t_2 \rangle \rightsquigarrow t_i : A_i\)}
\end{prooftree}

\begin{prooftree}
    \AxiomC{\(\Gamma \vdash t_0 : C\)}
    \AxiomC{\(\Gamma, x : \nat, y : C \vdash t_s (x, y) : C\)}
    \BinaryInfC{\(\Gamma \vdash \rec^C(t_0; (x, y, t_s); 0) : C \rightsquigarrow t_0 : C\)}
\end{prooftree} 

\begin{prooftree}
    \AxiomC{\(\Gamma \vdash m : \nat\)}
    \AxiomC{\(\Gamma \vdash t_0 : C\)}
    \AxiomC{\(\Gamma, x : \nat, y : C \vdash t_s (x, y) : C\)}
    \TrinaryInfC{\(\Gamma \vdash \rec^C(t_0; (x, y, t_s); S(m)) \rightsquigarrow t_s [m/x, \rec^C(t_0; (x, y, t_s); m)/y] : C\)}
\end{prooftree} 

Проведем теперь испытание наших определений и ``запрограммируем'' что-нибудь. Начнем с функции суммы. Как известно \(x + 0 = x\) и \(x + S y = S(x + y)\). Тогда мы можем в нашей системе типов определить функцию \(\mathrm{add}(x, y)\) как

\begin{prooftree}
    \AxiomC{\(x, y: \nat \vdash y: \nat\)}
    \AxiomC{\(x, y: \nat \vdash x : \nat\)}
    \AxiomC{\(x,y,z,w : \nat \vdash S(w) : \nat\)}
    \TrinaryInfC{\(x,y:\nat \vdash \rec(x; z, w, S(w); y) : \nat\)}
    \UnaryInfC{\(\mathrm{add}(x, y)\)}
\end{prooftree}

\begin{eg}
    \[\mathrm{add}(S0, 0) \equiv \rec(S0; z, w, S(w); 0) \rightsquigarrow S0,\]
    то есть \(1 + 0 = 1\).

    \begin{align*}\mathrm{add}(SS0, SS0) \equiv \rec(SS0; z, w, S(w); SS0) \rightsquigarrow Sw [\rec(SS0; z, w, S(w); S0)/w] \equiv \\ \equiv S(\rec(SS0; z, w, S(w); S0)) \rightsquigarrow S(Sw[\rec(SS0; z, w, S(w); 0)]/w) \equiv \\ \equiv SS(\rec(SS0; z, w, S(w); 0)) \rightsquigarrow SSSS(0),\end{align*}
    действительно \(2 + 2 = 4\).
\end{eg}

\section{Модель теории множеств для теории типов}
Мы будем использовать стандартную модель, где для каждого типа \(A\) существует множество \(\llbracket A\rrbracket^\mathcal{M}\), где \(\mathcal{M}\) --- наша модель. Мы будем её опускать, поскольку она стандартная. Далее мы определяем множества рекурсивно по типу \(A\):
\begin{itemize}
    \item \(\llbracket 1 \rrbracket := 1\) (или любой другой синглтон)
    \item \(\llbracket A \times B \rrbracket := \llbracket A \rrbracket \times \llbracket B \rrbracket\)
    \item \(\llbracket \N \rrbracket := \N\)
\end{itemize}

Для контекста \(\Gamma = (x_1 : A_1, \dots, x_n : A_n)\) мы определяем множество \(\llbracket \Gamma \rrbracket := \prod_i \llbracket A_i \rrbracket\). Далее нам требуется рекурсивно определить правила типизации. Это можно сделать двумя способами. 

Мы для любого контекста \(\Gamma\) и типа \(A\) мы определим функцию \(\llbracket \Gamma \rrbracket \underset{\llbracket \Gamma \vdash t : A \rrbracket}{\to} \llbracket A \rrbracket\). Она будет правильно определена тогда и только тогда, когда соблюдены правила типизации. Иначе мы получим \(\mathrm{undefined}\). Итак, будем действовать рекурсивно по терму \(t\):
\begin{itemize}
    \item \(\llbracket \Gamma \rrbracket \overset{\llbracket \Gamma \vdash x : A \rrbracket}{\longrightarrow} \llbracket A \rrbracket := \begin{cases}\llbracket \Gamma \rrbracket \overset{\pi_x}{\longrightarrow} \llbracket A \rrbracket & \text{если }(x:A) \in \Gamma,\\ \text{undefined} & \text{иначе}\end{cases}.\)

    \item \(\llbracket \Gamma \rrbracket \overset{\llbracket \Gamma \vdash \langle \rangle : A \rrbracket}{\longrightarrow} \llbracket A \rrbracket := \begin{cases}\llbracket \Gamma \rrbracket \overset{}{\longrightarrow} \llbracket A \rrbracket = 1 & \text{если }A=1,\\ \text{undefined} & \text{иначе}\end{cases}.\)

    \item \(\llbracket \Gamma \rrbracket \overset{\llbracket \Gamma \vdash \langle a, b \rangle^{B_1 \times B_2} : A \rrbracket}{\longrightarrow} \llbracket A \rrbracket := \begin{cases}\llbracket \Gamma \rrbracket \overset{\langle \llbracket \Gamma \vdash b_1 : B_1 \rrbracket, \llbracket \Gamma \vdash b_2 : B_2 \rrbracket \rangle}{\longrightarrow} \llbracket A \rrbracket = \llbracket B_1 \rrbracket \times \llbracket B_2 \rrbracket & \text{если }A=B_1 \times B_2,\\ \text{undefined} & \text{иначе}\end{cases}.\)

    \item \(\llbracket \Gamma \rrbracket \overset{\llbracket \Gamma \vdash \pi_i^{B_1, B_2} p : A \rrbracket}{\longrightarrow} \llbracket A \rrbracket := \begin{cases}\llbracket \Gamma \rrbracket \overset{\pi_i \llbracket \Gamma \vdash p : B_1 \times B_2 \rrbracket}{\longrightarrow} \llbracket A \rrbracket & \text{если }A = B_1 \times B_2 \in \Gamma,\\ \text{undefined} & \text{иначе}\end{cases}.\)

    \item \(\llbracket \Gamma \rrbracket \overset{\llbracket \Gamma \vdash 0 : A \rrbracket}{\longrightarrow} \llbracket A \rrbracket := \begin{cases}\llbracket \Gamma \rrbracket \overset{0}{\longrightarrow} \llbracket \N \rrbracket & \text{если }A=\N \in \Gamma,\\ \text{undefined} & \text{иначе}\end{cases}.\)

    \item \(\llbracket \Gamma \rrbracket \overset{\llbracket \Gamma \vdash S_n : A \rrbracket}{\longrightarrow} \llbracket A \rrbracket := \begin{cases}\llbracket \Gamma \rrbracket \overset{\llbracket \Gamma \vdash n : \N \rrbracket}{\longrightarrow} (\N \to \N) & \text{если }A = \N,\\ \text{undefined} & \text{иначе}\end{cases}.\)
\end{itemize}

\begin{thebibliography}{2}

    \bibitem{Shaph}
    Шафаревич И.Р.
    \newblock {\em Основы алгебраической геометрии.}
    \newblock УМН, 24:6(150) (1969), 3–184; Russian Math. Surveys, 24:6 (1969), 1–178.

    \bibitem{Harper}
    Robert Harper.
    \newblock {\em Type Systems for Programming Languages.}
    \newblock School of Computer Science, Carnegie Mellon University, Spring, 2000,
    \newblock url: \url{https://people.mpi-sws.org/~dreyer/ats/papers/harper-tspl.pdf}

    \bibitem{Lof}
    Per Martin-L\"{o}f.
    \newblock {\em Intuitionistic Type Theory.}
    \newblock Bibliopolis.
    \newblock url: \url{https://archive-pml.github.io/martin-lof/pdfs/Bibliopolis-Book-retypeset-1984.pdf}

    \bibitem{Smith}
    Bengt Nordstr\"{o}m, Kent Petersson, Jan M. Smith.
    \newblock {\em Programming in Martin-L\"{o}f's Type Theory.}
    \newblock Department of Computing Sciences, University of G\"{o}teborg, Sweden.

    \bibitem{}
    Аксель П. и др.
    \newblock {\em Гомотопическая теория типов.}
    \newblock Программа Унивалентных Оснований, Иститут Перспективных Исследований, пер.: Геннадий Чернышев,
    \newblock url: \url{https://henrychern.wordpress.com/wp-content/uploads/2022/10/hott2.pdf}

\end{thebibliography}

\end{document}
